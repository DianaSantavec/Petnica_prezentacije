\documentclass{beamer}
\usepackage{xcolor}
\usepackage{listings}

\usepackage{lmodern}
\usepackage[T1]{fontenc}

\usepackage{wasysym}



\usetheme[]{Madrid}

\title{Bugs, bugs everywhere}
\subtitle[]{\textit{ - Pregled nekih zanimljivih bagova -}}
\author[Diana Šantavec]{Diana Šantavec \\ \small \url{diana.santavec@gmail.com}}
\institute{Istraživačka stanica Petnica}
\titlegraphic{\includegraphics[width=2cm]{img/petnica.png}}
\date{07.04.2024.}


\lstset{language=C,keywordstyle={\bfseries \color{blue}}}

\setbeamertemplate{footline}
{
  \leavevmode%
  \hbox{%
  \begin{beamercolorbox}[wd=.333333\paperwidth,ht=2.25ex,dp=1ex,center]{author in head/foot}%
    \usebeamerfont{author in head/foot}\insertshortauthor
  \end{beamercolorbox}%
  \begin{beamercolorbox}[wd=.333333\paperwidth,ht=2.25ex,dp=1ex,center]{title in head/foot}%
    \usebeamerfont{title in head/foot}\insertshorttitle
  \end{beamercolorbox}%
  \begin{beamercolorbox}[wd=.333333\paperwidth,ht=2.25ex,dp=1ex,right]{date in head/foot}%
    \usebeamerfont{date in head/foot}\insertshortdate{}\hspace*{2em}
    \insertframenumber{} / \inserttotalframenumber\hspace*{2ex} 
  \end{beamercolorbox}}%
  \vskip0pt%
}

\begin{document}

\frame{\titlepage}


\begin{frame}
\frametitle{Uvod}
\begin{center}
    \includegraphics[width=10cm]{img/bugs_bugs_everywhere.png}
\end{center}
\end{frame}

%\begin{frame}
%    \frametitle{Literally everywhere}
%    \begin{center}
%        \includegraphics[width=12cm]{img/bus.png}
%    \end{center}
%\end{frame}

\begin{frame}
    \frametitle{Literally everywhere}
    \begin{center}
        \includegraphics[width=12cm]{img/a_lot_of_errors.png}
    \end{center}
\end{frame}

\begin{frame}
    \frametitle{Literally everywhere}
    \begin{center}
        \includegraphics[width=10cm]{img/train.png}
    \end{center}
\end{frame}

\begin{frame}
    \frametitle{Literally everywhere}
    \begin{center}
        \includegraphics[width=10cm]{img/enp.jpeg}
    \end{center}
\end{frame}

% Prilazi auto sa prikolicom. ENP tag se čuje, daje signal da je naplatio i da može da prođe
% Dešava se bug u backend-u i sistem zapravo nije naplatio, te nije podigao rampu
% Kako je vozač čuo signal, daje gas i ne stiže da stane, te udara u rampu

\begin{frame}
    \frametitle{Poreklo naziva}
    \begin{center}
        \includegraphics[width=5cm]{img/vacuum_tube.png}
        \includegraphics[width=5cm]{img/first_generation.png}
    \end{center}
    \begin{itemize}
        \item Računari prve generacije (1940-1960) \newline
        \item Vakuumske cevi
    \end{itemize}
\end{frame}

\begin{frame}
    \frametitle{Poreklo naziva}
    \begin{center}
        \includegraphics[width=8cm]{img/bug_in_notes.png}
    \end{center}
    \begin{itemize}
        \item Dr Grace Hopper (09.09.1947)
    \end{itemize}
\end{frame}

\section{Bagovi sa okruženjem}

\begin{frame}
\frametitle{printf ne radi?}
\begin{itemize}
    \item printf je funkcija za ispis teksta u C-u \newline
    \item \lstinline|pritnf("Hello world");| \newline
    \item Prelazak u novi red je kontrolni karakter \textbackslash n
\end{itemize}
\end{frame}

\begin{frame}
\frametitle{printf ne radi?}
\begin{itemize}
    \item \textbf{Situacija:} C program se pokreće u virtualnoj mašini i ispisuje poruku na terminal \newline
    \item \textbf{Problem:} Tekst iz programa se nasumično ne ispisuje
\end{itemize}
\end{frame}

\begin{frame}
\frametitle{printf ne radi?}
\begin{itemize}
    \item \textbf{Obrazloženje:} Linija terminala nije radila line wrap \newline
    \begin{center}
        \includegraphics[width=10cm]{img/terminal.png}
    \end{center}
    \item PS1 ili \lstinline|set horizontal-scroll-mode-off| \newline
    \item \textbf{Rešenja: }
    \begin{itemize}
        \item Dodavanje nove linije na kraju ispisa \newline
        \item Ispravljanje vrednosti env varijabli \newline
    \end{itemize}  
    
\end{itemize}
\end{frame}


\begin{frame}
\frametitle{Ali radi na mom računaru...}
\begin{itemize}
    \item \textbf{Situacija:} Kod pisan na Windows-u radi, ali kada se prebaci na Linux sistem i kompajlira, vraća grešku \newline
\end{itemize}
    \begin{center}
        \includegraphics[width=10cm]{img/LF_CRLF.png}
    \end{center}
\end{frame}

\begin{frame}
    \frametitle{Ali radi na mom računaru...}
    \begin{itemize}
        \item \textbf{Problem:} Različiti kraj linija \newline
        \item \textbf{Obrazloženje:} na Windows-u se koristi CRLF a na Linux-u LF
    \end{itemize}
    

\end{frame}

\begin{frame}
    \frametitle{Ali radi na mom računaru...}
    \begin{itemize}
        \item DOS/Windows: pratio mašine za pisanje \newline
        \item Unix: izbacio CR        \newline
    \end{itemize}
    \begin{center}
        \includegraphics[width=12cm]{img/hex_line_ending.png}
    \end{center}
    \end{frame}

\begin{frame}
    \frametitle{Ali radi na mom računaru...}    
    \begin{center}
        \includegraphics[width=12cm]{img/ascii_table.png}
    \end{center}
\end{frame}

\begin{frame}
    \frametitle{Ali radi na mom računaru...}
    \begin{itemize}
        \item \textbf{Situacija:} Iz logova se izvlači timestamp i poredi se da li je isti kao i očekivan \newline
        \item Poređenje se radi sa \textit{datetime} objekata, tako da se string pretvara u \textit{datetime} objekat \newline
        \item Kada se test pokrene na lokalnom računaru, prolazi, ali na serveru ne
    \end{itemize}
\end{frame}

\begin{frame}
    \frametitle{Ali radi na mom računaru...}
    \begin{itemize}
        \item \textbf{Problem:} Različiti formati datuma na samom operativnom sistemu \newline
        \item \textbf{Obrazloženje:} C\# kada konvertuje string u datum uzima regionalna podešavanja operativnog sistema \newline
    \end{itemize}
\end{frame}

\begin{frame}
\frametitle{Weird filenames making weird behaviours}
\begin{itemize}
    \item Komanda se sastoji od naziva komande i liste parametara razdvojenih razmakom \newline
    \item \lstinline|touch naziv1 naziv2 naziv3| \newline
    \item Nova linija označava kraj komande \newline
    \item \textbf{Situacija:} Pokrećemo komandu sa listom fajlova i dobijamo grešku da ne postoji komanda "naziv fajla" \newline
        
\end{itemize}
\end{frame}

\begin{frame}
    \frametitle{Wierd filemanes making weird behaviours}
    \begin{itemize}
        \item \textbf{Problem:} Ime fajla je bilo nova linija \newline
        \item \textbf{Obrazloženje: } \begin{enumerate}
            \item Linux dozvoljava da nazivi fajlova budu specijalni karakteri \newline
            \item Komanda može da čita listu parametara iz fajla
        \end{enumerate}
    \end{itemize}
\end{frame}

\begin{frame}
    \frametitle{Stari UNIX-ovi bagovi}
    \begin{itemize}
        \item \lstinline|/etc/passwords| \newline
        \item \lstinline|lpr| \newline
        \item \lstinline|setuid| proces koji ga pokrene ima sva prava \newline
        \item prilikom nasilnog prekidanja pravi se file core u koji se upisuje poruka o grešci \newline
        \item može se napraviti simbolički link na \lstinline|/etc/passwords|
    \end{itemize}
\end{frame}

\begin{frame}
    \frametitle{Linux - linkovi}
    \begin{itemize}
        \item \textbf{Situacija:} Ne može da se uloguje, login dugme stoji kao da je non stop pritisnuto, ali može da kuca druge stvari \newline
    \end{itemize}
\end{frame}

\begin{frame}
    \frametitle{Linux - linkovi}
    \begin{itemize}
        \item \textbf{Obrazloženje:} Postojali instalirani python2 i python3 \newline
        \item Kada se u terminalu ukuca python pokrece python2 \newline
        \item Pokušaj rešenja je bio da obriše python2 i napravi simbolički link na python kroz python3 \newline
        \item \textbf{Problem:} Sistem je koristio python2 za neke stvari \newline
        \item \textbf{Rešenje:} Ući u live sistem, obrisati link i instalirati python2 \newline
    \end{itemize}
\end{frame}

\begin{frame}
    \frametitle{Čudna poruka na ekranu}

    \begin{center}
        \includegraphics[width=12cm]{img/PC_Load_Letter.jpg}
    \end{center}
    

\end{frame}

\begin{frame}
    \frametitle{Čudna poruka na ekranu}
    \begin{itemize}
        \item HP LaseJet štampač \newline
        \item Poruka o grešci i statusu iz legacy razloga može da ima samo 2 karaktera (PC - Paper Cassette)\newline 
        \item Instrukcija ('LOAD') \newline
        \item Veličina papira (LETTER - US i Kanada)
    \end{itemize}
\end{frame}

\begin{frame}
    \frametitle{General Failure}
    \begin{center}
        \includegraphics[width=12cm]{img/general_failure_screen.jpg}
    \end{center}
\end{frame}

\begin{frame}
    \frametitle{General Failure}
    \begin{center}
        \includegraphics[width=12cm]{img/general_failure.jpg}
    \end{center}
\end{frame}

\begin{frame}
    \frametitle{The 500-mile email}
    \begin{itemize}
        \item \textbf{Situacija:} Podešen je novi mail server i kada se šalje mail na univerzitet koji je udaljen više od oko 520 milja, mail ne biva poslat
    \end{itemize}
    \vspace{1cm}
    \begin{center}
        \LARGE \textbf{Zašto?}
    \end{center}
\end{frame}

\begin{frame}
    \frametitle{The 500-mile email}
    \begin{itemize}
        \item Konfiguracija je pisana na novijem SunOS sa Sendmail8, a koristio se stari \newline
        \item Sendmail5 je imao predefinisanu vrednost vremena za čekanje od 0s i ukoliko u 0s ne dobije odgovor smatra je neuspešno slanje \newline
        \item Postoji mali prozor vremena, koji ukoliko server koji prima mail nije pod velikim load-om \newline
        \item Brzina svetlosti u optičkom kablu sa zakašnjenjem u ruteru i serveru daje oko 500 milja 
    \end{itemize}
\end{frame}

\begin{frame}
    \frametitle{Ali i u Petnici\dots}
    \begin{itemize}
        \item Na ruter je postavljena konfiguracija sa drugog rutera \newline
        \item Postojala je razlika u verzijama sistema na ruterima \newline
        \item Konfiguracija je bila za stariju verziju sistema i sve razlike u parametrima je sam dopunio \newline
        \item Problem kada je ruter postavljen je što je deo mreže radio, a deo nije, jer je deo konfiguracije padao
    \end{itemize}
\end{frame}

\begin{frame}
    \frametitle{HTTP kodovi za greške}
    \begin{itemize}
        \item 400 - Bad Request \newline
        \item 403 - Forbidden \newline
        \item 404 - Not Found \newline
        \item \textbf{418 - ?} \newline
        \item 451 - Unavailable For Legal Reasons \newline
        \item 500 - Internal Server Error \newline    
    \end{itemize}
\end{frame}

\begin{frame}
    \frametitle{HTTP kodovi za greške}
    \begin{itemize}
        \item 400 - Bad Request \newline
        \item 403 - Forbidden \newline
        \item 404 - Not Found \newline
        \item \textbf{418 - I'm a teapot} \newline
        \item 451 - Unavailable For Legal Reasons \newline
        \item 500 - Internal Server Error \newline 
    \end{itemize}
\end{frame}

\begin{frame}
    \frametitle{418 - I'm a teapot}
    \begin{itemize}
        \item HTCPCP: Protokol za kontrolu, monitoring i dijagnostifikovanje kuvala za kafu \newline
        \item Prvoaprilska šala \newline
        \item Emacs implementirao klijentsku stranu \newline
    \end{itemize}
\end{frame}

\begin{frame}
    \frametitle{Još bagova sa okruženjem?}
    \begin{itemize}
        \item Dirty cow \newline
        \item PwnKit \newline
        \item Y2K i Y2K38 problem \newline
        \item log4j \newline
    \end{itemize}
\end{frame}

\section{Bagovi sa tipovima}

\begin{frame}
    \frametitle{Bagovi sa tipovima}
    \begin{center}
        \begin{tabular}{ |c|c|c| }%{width}[pos]{cols}
            \hline
            Tip & Veličina u bajtovima & Opseg vrednosti \\
            \hline
            bool & 1 (8bit) & True / False \\
            \hline
            char & 1 (8bit) & od -128 do 127 \\
            \hline
            uint16\_t & 2 (16bit) & od 0 do 65535 \\
            \hline
            int32\_t & 4 (32bit) & od -2,147483,468 do 2,147483,469 \\            
            \hline
        \end{tabular}
    \end{center}
\end{frame}

\begin{frame}
    \frametitle{Bagovi sa tipovima}
    
    \begin{itemize}
        \item \textit {IEEE 754} \newline
        \item 32bit
         \begin{itemize}
            \item znak 1 bit
            \item eksponent 8 bitova
            \item mantisa (frakcija) 23 bita
        \end{itemize}
        \item 64bit
         \begin{itemize}
            \item znak 1 bit
            \item eksponent 11 bitova
            \item mantisa (frakcija) 52 bita
        \end{itemize}
    \end{itemize}

    \begin{center}
        \includegraphics[width=11cm]{img/32bit_ieee754.png}
    \end{center}
\end{frame}

\begin{frame}
    \frametitle{Greška "pas 38"}
    \begin{itemize}
        \item \textbf{Situacija: } Imena pasa se upisuju u bazu, ali ako ima više pasa sa istim nazivom doda im se broj 1,2,... \newline
        \item \textbf{Problem: } Kada u bazi postoji 37 pasa sa istim imenom, 38. ne može da se upiše \newline
        \item 00100110 (8bit)
    \end{itemize}
\end{frame}

\begin{frame}
    \frametitle{Greška "pas 38"}
    \begin{itemize}
        \item \textbf{Obrazloženje:} u bazi je polje char (6) a brojevi se zapisuju u formatu rimskih brojeva \newline
        \item XXXVIII
    \end{itemize}
\end{frame}

\begin{frame}
    \frametitle{Švajcarski vozovi duhovi}
    \begin{itemize}
        \item \textbf{Situacija:} Vozovi nestaju iz sistema \newline
        \item \textbf{Problem:} Brojač osovina je 8bitni broj (do 256) \newline
        \item \textbf{Obrazloženje:} Kada prođe voz sa tačno 256 osovina, desi se overflow i brojač se resetuje na 0, te se voz više ne vidi \newline
        \item \textbf{Rešenje: ?}
    \end{itemize}
\end{frame}

\begin{frame}
    \frametitle{Švajcarski vozovi duhovi}
    \begin{itemize}
        \item \textbf{Rešenje: } Promena regulativa, zabranjeni vozovi sa 256 osovina
    \end{itemize}
    \begin{center}
        \includegraphics[width=12cm]{img/ghost_train.png}
    \end{center}
\end{frame}

\begin{frame}
    \frametitle{Random is not random}
    \begin{center}
        \includegraphics[width=12cm]{img/Random Dilbert.jpg}
    \end{center}
\end{frame}

\begin{frame}
    \frametitle{Random is not random}
    \begin{itemize}
        \item Harverski random brojevi \newline
        \item Pseudorandom brojevi \newline
        \item Algoritam radi sa \textit{seed-om} \newline
        \item 
    \end{itemize}
\end{frame}

\begin{frame}
    \frametitle{Lavarand (Silicon graphics)}
    \begin{center}
        \includegraphics[width=12cm]{img/Lavarand.jpg}
    \end{center}
\end{frame}

\begin{frame}
    \frametitle{Press your luck}
    \begin{center}
        \includegraphics[width=10cm]{img/Press_your_luck.png}
    \end{center}
\end{frame}

\begin{frame}
    \frametitle{Press your luck}
    \begin{itemize}
        \item Vozač kamiona za sladoled Michael Larson - 110,237\$ (8 puta više od prosečnog igrača) \newline
        \item Prvo se odgovara na trivija pitanja i tako se zarađuju "spinovi" \newline
        \item Jako se kratko zadržava na slikama, te je odabir nasumičan \newline
    \end{itemize}
\end{frame}

\begin{frame}
    \frametitle{Press your luck}
    \begin{itemize}
        \item Zapravo je postojalo 5 predefinisanih krugova \newline
        \item Snimao epizode i uočio i zapamtio sekvencu \newline
    \end{itemize}
\end{frame}

\begin{frame}
    \frametitle{NULL - "billion-dollar mistake"}
    \begin{itemize}
        \item null pointer exception \newline
        \item Tony Hoare uvodi NULL u ALGOL W (1965)\newline
        \item Naslednik ALGOL 60 (1960) \newline
        \item Konferencija 2009.
    \end{itemize}
\end{frame}
    
\begin{frame}
    \frametitle{NULL - "billion-dollar mistake"}
    \begin{center}
        \includegraphics[width=6.5cm]{img/ALGOL60.png}    
    \end{center}
\end{frame}

\begin{frame}
    \frametitle{NULL - "bilion-dollar mistake"}
    \begin{itemize}
        \item \textbf{Situacija: }
        \begin{itemize}
            \item ALGOL W uvodi RECORD (class/struct) \newline
            \item Potrebna referenca na stack-u na RECORD \newline
            \item Uvodi se NULL, kada ne postoji podatak na heap-u
        \end{itemize}
    \end{itemize}
\end{frame}

\begin{frame}[fragile]
    \frametitle{NULL - "billion-dollar mistake"}
    \begin{lstlisting}
        RECORD PERSON (
            STRING(25) NAME;
            INTEGER AGE;
            REFERENCE (PERSON) FATHER, MOTHER
        );
        // pretpostavimo da postoji referenca R 
        // koja pokazuje na neku osobu
        REFERENCE(PERSON)P, M;
        P := FATHER(FATHER(R));
        IF P = NULL THEN
            M := MOTHER(P)
        ELSE
            P
    \end{lstlisting}
    \begin{itemize}
        \item \textit{\href{https://en.wikipedia.org/wiki/ALGOL_W}{Link ka referenci}}
    \end{itemize}
\end{frame}

\begin{frame}
    \frametitle{Steve Null}
    \begin{itemize}
        \item HR unese zaposlenog i on nestane iz tabele \newline
        \item Problem sa prezimenom
    \end{itemize}
\end{frame}

\begin{frame}[fragile]
    \frametitle{Najskuplja '-'}
    
    \begin{itemize}
        \item Mariner 1 - 1960thi NASA \newline
        \item Matematičke formule se pretvarale u kod \newline
        \item $\overline{R}$ "smoth over period of time" \newline
        \item Ali greškom upisano $R$ \newline
        \item 18.000.000\$
    \end{itemize}
\end{frame}

\section*{Bagovi u svemiru}
\begin{frame}
    \frametitle{Greška u orbiti Marsa}
    \begin{itemize}
        \item Mars Climate Orbiter (1998) \newline
        \item \textbf{Situacija:} Orbiter prilazi Marsu i odjednom pada \newline
        \item \textbf{Problem: } Različite merne jedinice (mm i in) \newline
        \item Lockheed Martin koristio inče \newline
        \item 327.000.000\$
    \end{itemize}
\end{frame}

\begin{frame}
    \frametitle{Greška u orbiti Marsa}
    \begin{center}
        \includegraphics[width=10cm]{img/mars.png}
    \end{center}
\end{frame}

\begin{frame}
    \frametitle{Ariane 5}
    \begin{itemize}
        \item European Space Agency (1996)\newline
        \item Floating point bug - integer overflow \newline
        \item \textbf{Situacija:} Raketa je nedugo nakon uzletanja skrenula i eksplodirala \newline
        \item 500.000.000\$
    \end{itemize}
\end{frame}

\begin{frame}
    \frametitle{Ariane  5}
    \begin{itemize}
        \item \textbf{Problem:} 64bitna floating-point vrednost je konvertovana u 16bitni označeni integer \newline
        \item Broj koji predstavlja horizontalnu brzinu je bio veći od 32767 \newline
        \item Debugging podaci iscureli u memoriju za navigaciju \newline
        \item Backup računar je imao isti kod \newline
        \item Ili? Sistem za samouništenje se upalio
    \end{itemize}
\end{frame}

\begin{frame}
    \frametitle{Ariane 5}
    \begin{center}
        \includegraphics[width=12cm]{img/ariane5.png}
    \end{center}
\end{frame}

\begin{frame}
    \frametitle{Slični bagovi}
    \begin{itemize}
        \item Boeing 737 MAX \newline
        \item Knight Capital \newline
        \item Therac-25 \newline
    \end{itemize}
\end{frame}

\begin{frame}
    \frametitle{Bagovi u igricama}
    \begin{itemize}
        \item Sa kakvim bagovima ste se susretali?
    \end{itemize}
\end{frame}

\section{Bagovi u igricama}
\begin{frame}
    \frametitle{Intentional bugs}
    \begin{itemize}
        \item Grey-box testing \newline
    \end{itemize}
    \begin{center}
        \includegraphics[width=11cm]{img/intentional_bugs.png}
    \end{center}
\end{frame}

\begin{frame}
    \frametitle{Minecraft}
    \begin{center}
        \includegraphics[width=12cm]{img/farlands.jpg}
    \end{center}
\end{frame}

\begin{frame}
    \frametitle{Minecraft}
    \begin{center}
        \includegraphics[width=8cm]{img/farlands_grid.png}
    \end{center}
\end{frame}

\begin{frame}
    \frametitle{Minecraft}
    \begin{itemize}
        \item Sky far lands \newline
        \item Void far lands \newline
        \item Far lands \newline
        \item Vertex far lands \newline
    \end{itemize}
\end{frame}

\begin{frame}
    \frametitle{Minecraft}
    \begin{itemize}
        \item 12.550.824 \newline
        \item Generisanje terena - Perlin noise \newline
        \item Generisanje random brojeva, ali je teren "gladak" \newline
        \item 16 octaves
    \end{itemize}
\end{frame}

\begin{frame}
    \frametitle{Minecraft}
    \begin{itemize}
        \item 171.103 pixela na mapi predstavljaju jedan blok \newline
        \item $12.550.824 = 2^{31} / 171.103$ \newline
        \item integer overflow \newline
        \item predmeti se drugačije ponašaju \newline
        \item problem sa zvukom
    \end{itemize}
\end{frame}

\begin{frame}
    \frametitle{Wing Commander}
    \begin{center}
        \includegraphics[width=12cm]{img/wing_commander.png}
    \end{center}
\end{frame}

\begin{frame}
    \frametitle{Wing Commander}
    \begin{center}
        \includegraphics[width=12cm]{img/wing_commander.png}
    \end{center}
    \begin{itemize}
        \item Poruka greške pretvorena u pozdravnu poruku
    \end{itemize}
\end{frame}

\begin{frame}
    \frametitle{Igrica Force 21}
    \begin{itemize}
        \item Kamera prestane da prati igrača \newline
        \item Nasleđuje klasu PhysicalObject \newline
        \item Trpi damage i biva "ubijena"
    \end{itemize}
\end{frame}

\section*{}
\begin{frame}
    \frametitle{Dodatni saveti}
    \begin{itemize}
        \item Nazivi fajlova i poruke u programu \newline
        \item Ne kopirati komande/kod bez razumevanja \newline
        \item Debugger-i su prijatelji \smiley
    \end{itemize}
\end{frame}

\begin{frame}
    \frametitle{Dodatni saveti}
    \begin{itemize}
        \item Pisanje testova \newline
        \item Rekurzija \newline
        \item Endianness
    \end{itemize}
\end{frame}


\begin{frame}
    \frametitle{Reference}
    \begin{itemize}
        \item Kolege \smiley \newline
        \item \href{https://youtu.be/qC_ioJQpv4E}{Programming's Greatest Mistakes - Mark Rendle - NDC Copenhagen 2022} \newline
        \item \href{https://youtu.be/6xrGo1IIB3w}{FAIL - Kevlin Henney - GOTO 2022} \newline
        \item Dirty programming, Aleksandar Beserminji \newline
        \item \href{https://youtu.be/M_iiXaaF5T4}{I made a Game with Intentional Bugs} \newline
        \item \href{https://youtu.be/4f5M1fHxRrQ}{Minecraft - Farlands}
    \end{itemize}
    \end{frame}


\begin{frame}
    \frametitle{Diskusija}
    \begin{itemize}
        \item Kakve ste bagove sretali?
    \end{itemize}
\end{frame}

\begin{frame}
    \frametitle{HVALA NA PAŽNJI!}
    \begin{center}
        \Huge Pitanja?    
    \end{center}
\end{frame}

\end{document}