\documentclass{beamer}
\usepackage{xcolor}
\usepackage{listings}

\usepackage{lmodern}
\usepackage[T1]{fontenc}

\usepackage{wasysym}



\usetheme[]{Antibes}

\title{Konkurentno programiranje}
%\subtitle[]{\textit{ - Pregled nekih zanimljivih bagova -}}
\author[Diana Šantavec]{Diana Šantavec \\ \small \url{diana.santavec@gmail.com}}
\institute{Istraživačka stanica Petnica}
\titlegraphic{\includegraphics[width=2cm]{img/Petnica.png}}
\date{25.04.2023.}


\lstset{language=C,keywordstyle={\bfseries \color{blue}}}

\setbeamertemplate{footline}
{
  \leavevmode%
  \hbox{%
  \begin{beamercolorbox}[wd=.333333\paperwidth,ht=2.25ex,dp=1ex,center]{author in head/foot}%
    \usebeamerfont{author in head/foot}\insertshortauthor
  \end{beamercolorbox}%
  \begin{beamercolorbox}[wd=.333333\paperwidth,ht=2.25ex,dp=1ex,center]{title in head/foot}%
    \usebeamerfont{title in head/foot}\insertshorttitle
  \end{beamercolorbox}%
  \begin{beamercolorbox}[wd=.333333\paperwidth,ht=2.25ex,dp=1ex,right]{date in head/foot}%
    \usebeamerfont{date in head/foot}\insertshortdate{}\hspace*{2em}
    \insertframenumber{} / \inserttotalframenumber\hspace*{2ex} 
  \end{beamercolorbox}}%
  \vskip0pt%
}

\begin{document}

\frame{\titlepage}

\begin{frame}
    \frametitle{Sadržaj}
    \begin{itemize}
        \item Podsećanje \newline
        \item Niti \newline
        \item Problemi i rešenja \newline
        \item 
    \end{itemize}
\end{frame}

\begin{frame}
    \frametitle{Podsećanje}
    \begin{itemize}
        \item Šta je proces? \newline
        \item Šta su niti? \newline 
    \end{itemize}
\end{frame}

\begin{frame}
    \frametitle{Niti}
    \begin{itemize}
        \item Zašto su niti bitne?
    \end{itemize}
\end{frame}

\section*{Niti}
\begin{frame}
    \frametitle{Pojam}
    \begin{itemize}
        \item \textit{lightweight process} \newline
        \item Nezavisni tokovi \newline
        \item Isti adresni prostor \newline
        \item Planer procesa ima kontrolu
    \end{itemize}

\end{frame}

\begin{frame}
    \frametitle{Memorija}
    \begin{itemize}
        \item Dele u okviru procesa: \begin{itemize}
            \item Data segment \newline
            \item Code segment \newline
            \item Fajlovi \newline
        \end{itemize}
        \item Imaju nezavisno: \begin{itemize}
            \item Stek (Instruction pointer i registri)\newline
        \end{itemize}
    \end{itemize}   
    %Text segment (instructions)
    %Data segment (static and global data)
    %BSS segment (uninitialized data)
    %Open file descriptors
    %Signals
    %Current working directory
    %User and group IDs
\end{frame}

\begin{frame}
    \frametitle{Memorija}
    \begin{center}
        \includegraphics[width=12cm]{img/process_vs_thread}
    \end{center}
\end{frame}

\begin{frame}
    \frametitle{Memorija}
    \begin{itemize}
        \item Komunikacija je brža nego između procesa \newline
        \item Zahtevaju manje resursa \newline
        \item Brže se prave (ne zahtevaju memory map, a ni sve resurse)
        \item Context switching je brži
    \end{itemize}
\end{frame}

\begin{frame}
    \frametitle{Podaci koje čuva OS}
    \begin{itemize}
        \item Thread ID \newline
        \item Saved registers, stack pointer, instruction pointer \newline
        \item Stack (local variables, temporary variables, return addresses) \newline
        \item Signal mask \newline
        \item Priority (scheduling information)
    \end{itemize}
\end{frame}

\begin{frame}
    \frametitle{Tipovi}
    \begin{itemize}
        \item Single-threaded \newline
        \item Multithreading \newline
        \item Kada se blokira neka nit, sve niti nastale od nje bivaju blokirane
    \end{itemize}
\end{frame}

\begin{frame}
    \frametitle{Tipovi}
    \begin{itemize}
        \item User level thread \newline
        \item Kernel-level thread
    \end{itemize}

    

\end{frame}

\begin{frame}
    \frametitle{Pravljenje niti}

    

\end{frame}

\begin{frame}
    \frametitle{Problemi}
    \begin{center}
        Kakvi problemi mogu nastati sa multithreading procesima?
    \end{center}
\end{frame}

\begin{frame}
    \frametitle{Problemi}
    \begin{center}
        Šta je deadlock?
    \end{center}
\end{frame}

\begin{frame}
    \frametitle{Pristup memoriji}
    \begin{itemize}
        \item Imamo dva procesa i pokušavaju da otvore jedan fajl \newline
        \item Imamo dve niti i pokušavaju da pristupe jednoj javnoj varijabli
    \end{itemize}
\end{frame}

\begin{frame}
    \frametitle{Problemi}
    \begin{itemize}
        \item Da li sledeći kod može da radi?
        % Kod koji pravi dve niti i bez provere cita i uvecava variablu
    \end{itemize}
\end{frame}

\begin{frame}
    \frametitle{Da li sledeći kod može da radi?}
    % dodavanje "lock" zastavice i provera da li je zaključano
\end{frame}

\begin{frame}
    \frametitle{Semafor}
    \begin{center}
        \includegraphics[width=5cm]{img/semaphore_trafic.png}
    \end{center}
\end{frame}

\begin{frame}
    \frametitle{Semafor}
    \begin{itemize}
        \item Omogućava da ne može više procesa istovremeno da koristi neki resurs \newline
        \item Operacije: \begin{itemize}
            \item Čekaj
            \item Signal
        \end{itemize}
    \end{itemize}
\end{frame}

\begin{frame}
    \frametitle{Semafor}
    %Opis sta su
\end{frame}

\begin{frame}
    \frametitle{Semafor}
    % Primer koda    

\end{frame}

\begin{frame}
    \frametitle{Mutex}
    % Opis sta su
\end{frame}

\begin{frame}
    \frametitle{Mutex}
    %Primer koda
\end{frame}

\begin{frame}
    \frametitle{Hardverska pomoć}
    %nesto
\end{frame}

\begin{frame}
    \frametitle{Monitori}
    %opis sta su

\end{frame}

\begin{frame}
    \frametitle{Problem 5 filozofa - podsećanje}

\end{frame}



\begin{frame}
    \frametitle{Five philosopher problem}
    \begin{itemize}
        \item Pet filozofa sedi za okruglim stolom i na smenu rezmišljaju i jedu. 
        Svaki ima ispred sebe tanjir sa špagetama i između svaka dva tanjira se nalazi viljuška.
         Da bi mogao da jede, filozofu trebaju dve viljuške. \newline
        \item Zastoj (deadlock)
    \end{itemize}
\end{frame}

\begin{frame}
    \frametitle{Five philosopher problem}
    \begin{center}
        \large{Demonstracija i rešenje}
    \end{center}
\end{frame}

\begin{frame}
    \frametitle{Problem proizvođača i potrošača}
    

\end{frame}


\begin{frame}
    \frametitle{Problem uspavanog berberina}

    

\end{frame}

\begin{frame}
    \frametitle{HVALA NA PAŽNJI!}
    \begin{center}
        \Huge Pitanja?    
    \end{center}
\end{frame}

\end{document}